
\documentclass[12pt]{article}
\usepackage[utf8]{inputenc}
\usepackage[brazil]{babel}
\usepackage{amsmath, amssymb}
\usepackage{graphicx}
\usepackage{booktabs}
\usepackage{geometry}
\usepackage{hyperref}
\usepackage{longtable}
\usepackage{caption}
\geometry{a4paper, margin=2.5cm}

\title{Análise de Similaridade Textual para Identificação de Crimes de Roubo em Registros Policiais}
\author{}
\date{}

\begin{document}

\maketitle

\section{Introdução}

O combate à criminalidade urbana permanece como um dos principais desafios enfrentados por autoridades e pesquisadores da área de segurança pública. Identificar e analisar delitos específicos, como o roubo, é essencial para compreender padrões, direcionar recursos de forma estratégica e desenvolver políticas eficazes. 

Com os avanços no Processamento de Linguagem Natural (PLN) e em técnicas de aprendizado de máquina, modelos computacionais agora podem ser aplicados para auxiliar na identificação de crimes semelhantes com base em suas descrições textuais.

\section{Objetivo do Estudo}

Este estudo tem como propósito principal reconhecer e categorizar registros criminais relacionados ao roubo, tendo como foco a palavra-chave \textit{``theft''}, presente em descrições textuais de delitos. O método proposto emprega técnicas de similaridade textual para identificar descrições que sejam idênticas ou semanticamente próximas à palavra \textit{``theft''}, mesmo que o termo exato não esteja presente.

\section{Metodologia e Implementação}

Para atingir esse objetivo, foi desenvolvido um código em Python utilizando bibliotecas amplamente reconhecidas na ciência de dados e aprendizado de máquina, como \texttt{pandas}, \texttt{scikit-learn} e \texttt{tabulate}. O processo seguiu as seguintes etapas:

\begin{itemize}
    \item \textbf{Importação e validação dos dados:} O conjunto de dados foi carregado a partir do arquivo \texttt{major-crime-indicators.csv}, assegurando-se da presença das colunas essenciais \texttt{OFFENCE} e \texttt{MCI\_CATEGORY}.
    \item \textbf{Pré-processamento textual:} As descrições de crimes na coluna \texttt{OFFENCE} foram padronizadas, com a remoção de valores nulos.
    \item \textbf{Conversão de texto em vetores:} O modelo TF-IDF (\textit{Term Frequency-Inverse Document Frequency}) foi empregado para transformar as descrições em representações numéricas que refletem a importância relativa das palavras.
    \item \textbf{Cálculo da similaridade:} A similaridade do cosseno foi utilizada para comparar a representação vetorial da palavra \textit{``theft''} com as demais descrições de crimes no conjunto de dados.
    \item \textbf{Classificação dos resultados:}
    \begin{itemize}
        \item Correspondência exata (similaridade = 1.0), indicando descrições idênticas a \textit{``theft''}.
        \item Alta similaridade (0.2 $\leq$ similaridade < 1.0), indicando descrições semanticamente próximas.
    \end{itemize}
    \item \textbf{Apresentação das análises:} Os resultados foram organizados em tabelas, destacando as descrições mais relevantes conforme a similaridade calculada.
\end{itemize}

\section{Resultados}

\subsection{Ocorrências com Similaridade Total (1.0)}

As descrições abaixo são idênticas à palavra-chave \textit{``theft''}, indicando registros diretamente associados a esse tipo de crime:

\begin{table}[h!]
\centering
\caption{Descrições com similaridade total}
\begin{tabular}{ll}
\toprule
\textbf{OFFENCE} & \textbf{MCI\_CATEGORY} \\
\midrule
Theft Over & Theft Over \\
Theft Over & Theft Over \\
Theft Over & Theft Over \\
Theft Over & Theft Over \\
\bottomrule
\end{tabular}
\end{table}

\subsection{Ocorrências com Alta Similaridade (0.2 $\leq$ similaridade < 1.0)}

Embora não correspondam exatamente ao termo \textit{``theft''}, estas descrições compartilham um contexto semântico fortemente relacionado:

\begin{table}[h!]
\centering
\caption{Descrições com alta similaridade semântica}
\begin{tabular}{lll}
\toprule
\textbf{OFFENCE} & \textbf{MCI\_CATEGORY} & \textbf{Similaridade Coseno} \\
\midrule
Theft Of Motor Vehicle & Auto Theft & 0.5586 \\
Theft From Motor Vehicle Over & Theft Over & 0.5586 \\
Theft Over - Shoplifting & Theft Over & 0.3362 \\
Theft Over - Bicycle & Theft Over & 0.2913 \\
Theft Over - Distraction & Theft Over & 0.2875 \\
Theft From Mail / Bag / Key & Theft Over & 0.2253 \\
Theft Of Utilities Over & Theft Over & 0.2183 \\
\bottomrule
\end{tabular}
\end{table}

\section{Conclusão}

A aplicação de técnicas de análise textual permitiu identificar com precisão as ocorrências criminais relacionadas ao roubo dentro do conjunto de dados analisado. A combinação do modelo TF-IDF com a similaridade do cosseno demonstrou-se eficaz tanto na detecção de descrições exatas quanto na identificação de variações semânticas associadas ao termo \textit{``theft''}.

A abordagem apresentada pode ser expandida para investigações mais abrangentes, incorporando padrões espaciais, temporais e possíveis correlações entre diferentes tipos de delitos, contribuindo para estratégias mais eficientes de combate à criminalidade.

\end{document}
